\documentclass[twoside]{article}
\usepackage{a4wide}
%\usepackage[dutch]{babel}
\usepackage{graphicx}

\raggedbottom
\parindent=0pt
\parskip=3pt

\newenvironment{itize}{\begin{list}{$\bullet$}{\parsep=0pt\parskip=0pt\topsep=0pt\itemsep=0pt}}{\end{list}}
\newenvironment{subitize}{\begin{list}{$-$}{\parsep=0pt\parskip=0pt\topsep=0pt\itemsep=0pt}}{\end{list}}
\newcommand{\simplepicture}[2]{\centerline{\mbox{\includegraphics[scale=#1]{figs/#2}}}}

\def\uncatcodespecials{\def\do##1{\catcode`##1=12 }\dospecials}
\def\setupverbatimp{\parskip=0pt\par\small\tt\obeylines\uncatcodespecials\catcode`\^=0\parskip=0pt\obeyspaces}
{\obeyspaces\global\let =~}
\def\prog{\par\sloppy\begingroup\begin{list}{}{\leftmargin=1cm\parskip=0pt\topsep=6pt}\item\setupverbatimp\doverbatimp}
{\catcode`\|=0 \catcode`\\=12 %
|obeylines|gdef|doverbatimp^^M#1\gorp{#1|end{list}|vspace{-3pt}|endgroup}}


\begin{document}
%\renewcommand{\textfraction}{0.1}



\title{UUAGC v.0.9.5\\Release notes}

\author{Arie Middelkoop}

\date{May 25th, 2007}

\maketitle


This note describes the differences of UUAGC release 0.9.5,
as compared to version 0.9.4 of April 2007.
Version 0.9.5 adds higher order attributes as new feature to the AG system.


\section{Introduction}

Higher Order Attribute Grammars (HAGs) allow the structure of the tree to depend on the semantics of the tree. Trees may be computed as attributes and those trees can be attributed as part of the whole tree. An attribute representing such a tree is called a Higher Order Attributes\footnote{Higher order attributes are also referred to as non-terminal attributes in literature.}. A higher order attribute is a local attribute $a$ of alternative $C$ with the following properties:

\begin{itize}
\item the value $v$ of attribute $a$ has type $N$
\item $N$ is a non-terminal
\item $v$ is attributed according to $N$
\item The inherited and synthesized attributes of $N$ can be accessed using field $a$ in $C$
\end{itize}

Note that $a$ is added to the end of the list of children of $C$ and participates as child (not as local attribute) in the automatic rule generation process (copy rules, etc).


\section{Syntax}

A higher order attribute is defined as a regular local attribute, except that it uses the field name $inst$ instead of $loc$, and it requires a type signature of a known non-terminal. Example:

\prog
DATA T
  | C

DATA N
  | A x : Int

ATTR T [ y : Int | | z : Int ]

SEM T
  | C  inst.a : N           -- obligatory type signature of a
       inst.a = N_A @lhs.y  -- add a child a of non-terminal N to C
       a.p    = 1           -- define inherited attribute p of N
       lhs.z  = @a.q        -- use synthesized attribute q of N

ATTR N [ p : Int | | q : Int ]

SEM N
  | A  lhs.q = @lhs.p + @x
\gorp


\section{Discussion}

\begin{itemize}
\item The cycle detection does not allow a higher order attribute to depend on its own synthesized attributes:
\prog
DATA Root
  | Root

DATA N
  | A

ATTR N [ | | s : N ]

SEM Root
  | Root  inst.x : N
          inst.y : N
          inst.x = @y.s  -- cycle
          inst.y = @x.s  -- cycle

SEM N
  | A  lhs.s = N_A
\gorp
\item Local and inst attributes live in different namespaces:
\prog
DATA T
  | C

ATTR T [ | | a : Int ]
SEM T
  | C  loc.a = 3
  
       inst.a : T
       inst.a = C  -- 'inst.a' itself is never selected during
                   --   the copy-rule process
       loc.b = @a + @loc.a  -- both refer to 'loc.a = 3'
\gorp
\item Higher order attributes are not stored in data types, neither in SELF attributes.
\item It is allowed to use an inst attribute on the right-hand side of an equation to represent its tree-value (not its semantics). Example:
\prog
DATA T
  | C

DATA InfList
  | Cons  next : InfList

SEM T
  | C  inst.a : InfList
       inst.a = InfList_Cons @inst.a
       a.nr   = 0

ATTR InfList [ nr : Int | | ]
SEM InfList
  | Cons  next.nr = @lhs.nr + 1
\gorp
\end{itemize}

\section{Additional compiler flags}

\begin{itemize}
\item --strictcase, force evaluation of the scrutinee of cases (in generated code, visit functions only)
\item --splitsems, split semantic function in smaller chunks
\item Added first and last as generic field names
\end{itemize}

\end{document}
