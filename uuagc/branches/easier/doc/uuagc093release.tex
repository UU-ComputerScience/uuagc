\documentclass[twoside]{article}
\usepackage{a4wide}
%\usepackage[dutch]{babel}
\usepackage{graphicx}

%\topmargin=-15mm
%\textheight=240mm
%\textwidth=150mm
%\oddsidemargin=6mm
%\evensidemargin=6mm
%\marginparwidth=12mm
%\marginparsep=6mm
\raggedbottom
\parindent=0pt
\parskip=3pt

\newenvironment{itize}{\begin{list}{$\bullet$}{\parsep=0pt\parskip=0pt\topsep=0pt\itemsep=0pt}}{\end{list}}
\newenvironment{subitize}{\begin{list}{$-$}{\parsep=0pt\parskip=0pt\topsep=0pt\itemsep=0pt}}{\end{list}}
\newcommand{\simplepicture}[2]{\centerline{\includegraphics[scale=#1]{figs/#2}}}

\def\uncatcodespecials{\def\do##1{\catcode`##1=12 }\dospecials}
\def\setupverbatimp{\parskip=0pt\par\small\tt\obeylines\uncatcodespecials\catcode`\@=0\parskip=0pt\obeyspaces}
{\obeyspaces\global\let =~}
\def\prog{\par\sloppy\begingroup\begin{list}{}{\leftmargin=1cm\parskip=0pt\topsep=6pt}\item\setupverbatimp\doverbatimp}
{\catcode`\|=0 \catcode`\\=12 %
|obeylines|gdef|doverbatimp^^M#1\gorp{#1|end{list}|vspace{-3pt}|endgroup}}


\begin{document}


\title{UUAGC v.0.9.3\\Release notes and Implementation notes}
\author{Jeroen Fokker}
\date{November 24th, 2006}
\maketitle

This report describes the differences of UUAGC version 0.9.3,
as compared to version 0.9.1 of December 2005.

In short, this version has:
\begin{itemize}
\item {\bf Better input}\\
	\begin{subitize}
	\item It was already possible to later add new alternatives to exisiting datatypes.
	      But now you can also later add new children to existing alternatives.
	\item You can also add children generically to multiple datatypes.
	\item It was already possible to declare an attribute for multiple datatypes.
	      But now you can also generically {\em define} these attributes in a single definition.
	\end{subitize}	      
\item {\bf Better output}
	\begin{subitize}
	\item The output is now ghc-6.6 compliant, because the lhs of a definition is no longer put in parentheses.
	\item There are generally less superfluous parentheses in the output,
	      which makes the generated code easier to read.
	\item Comments can be generated not only listing the children, but also the attributes and local variables.
	\item Error messages have a better layout.
	\end{subitize}
\item {\bf Better source}
	\begin{subitize}
	\item The five stages of processing that together form the main dataflow are made more uniform,
	      treating their input (tree and options) and output (next tree, errors, and additional strings)
	      in a more consistent way. This makes the source easier to understand and modify in the future.
	\item In imported Haskell ibraries, all needed functions are explicitly enumerated.
	      This makes it more transparent why a module is actually needed.
	\item Some code (especially the gathering of alle distributed information, and the generation of default rules)
	      has been rewritten in order to make it easier to understand.
    \end{subitize}
\end{itemize}

In section~1 we will introduce the new features of the input language.
In section~2 we describe the overall architecture of the program.
In section~3 we list the modifications that were made for this version.

\newpage

\section{New features}

\subsection{Add children to existing datatypes}

When defining a datatype in Haskell, you have to specify all alternatives in one declaration.
In contrast, in UUAGC it is possible to add new alternatives to existing datatypes.
This was already possible in earlier versions. For example:
\prog
-- initial definition
DATA Foo
   | One  a1 : {Int}
          a2 : {Int}
   | Two  b1 : {Int}
--other definitions
DATA Bar
   | First d : {Int}
-- add-on to first definition
DATA Foo
   | Three c1 : {Int}
\gorp
In the new version, it is also possible to add new children to existing alternatives.
For example, to add a second child to alternative \verb"Two" of datatype \verb"Foo",
you can extend the example above by:
\prog
-- add new child to existing alternative
data Foo
   | Two  b2 : {Int}
\gorp
In earlier versions, this would result in a `duplicate alternative' error,
but now it is allowed. Of course, the name of the added child \verb"b2" should differ
from the names of the existing child \verb"b1".


\subsection{Generically add children to multiple datatypes}

If two datatypes share a common part, it is now possible to write it only once,
and later extend it in different ways. For example:
\prog
-- common part
DATA Foo Bar
   | One a : {Int}
   | Two b : {Int}
-- two extensions
DATA Foo
   | Three c : {Int}
DATA Bar
   | Four  d : {Char}
\gorp
In earlier versions, it was a syntax error to enumerate more than one type name in a DATA header.
Even an attempt to capture both names in a set would fail in earlier versions:
\prog
SET Common = Foo Bar
DATA Common
   | One a : {Int}
   | Two b : {Int}
\gorp
It would throw a `duplicate synonym' error in earlier versions, but it is possible in the new version.


\subsection{Generically define attribute values}

In earlier versions, it was already possible to declare an attribute for multiple datatypes:
\prog
ATTR Foo Bar [ | | s : {Int} ]
\gorp
But it was not possible to {\em define} the {\em value} of the attribute generically.
In the new version, it is possible to have a single definition for two datatypes,
provided that they have a common alternative.
Of course, this situation will occur more often if the construction in the previous subsection
is used frequently.
We are now able to define in a single definition:
\prog
SEM Foo Bar
  | One  lhs.s = 3
\gorp
In earlier versions, it was a syntax error to enumerate more than one type name in a SEM definition,
and an apporach using SETs would also fail.


\subsection{Using wildcards}

In earlier versions, it was already possible to use a wildchard name \verb"*", and a name set difference operator \verb"-",
to generically define an attribute for more than one alternative. For example:
\prog
SEM Foo
  | * - One  lhs.s = 4
 \gorp
would define the attirbute for alternatives \verb"Two" and \verb"Three".

It should be noted that in the new version, this notation has an even more generic meaning:
\prog
SEM Foo Bar
  | * - One  lhs.s = 4
\gorp
will define the attribute for alternatives \verb"Two" and \verb"Three" of datatype \verb"Foo",
{\em and} for alternatives \verb"Two" and \verb"Four" of datatype \verb"Bar".

It is even allowed to use wildcards in the header of a SEM-definition, as in:
\prog
SEM *
  | * - One  lhs.s = 4
\gorp
Also, it is now allowed to use wildcards in the header of DATA and ATTR definitions.
Thus, we can add an uniform alternative to all existing datatypes, for example to provide 
a placeholder for error situations:
\prog
DATA *
  | Error
\gorp
And it is possoble to uniformly add a child to all alternatives of a (set of) datatype(s),
for example to store the location in the source file in all alternatives of both 
the \verb"Expr" and the "Stat" datatype:
\prog
DATA Expr Stat
  | *  pos : {Int}
\gorp





\section{Program architecture}





\section{Modifications}







\end{document}
